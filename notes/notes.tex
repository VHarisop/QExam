\documentclass[a4paper]{article}

\usepackage[margin=1in]{geometry}
\usepackage{../latex-macros/macros/latex-macros}
\usepackage{subcaption}

% Several math packages to use
\usepackage{amsmath, amssymb, amsthm}
\usepackage{bm, cancel}

\usepackage[boxruled]{algorithm2e}
\usepackage{caption}
\newenvironment{code}{\captionsetup{type=listing}}{}

% Include pgfplots / tikz options here
% PStricks package
\usepackage{pstricks}
\usepackage{pstricks-add}
\usepackage{tikz}
\usepackage{pgfplots}
\usetikzlibrary{
    matrix,chains,positioning,decorations.pathreplacing,arrows,patterns}
\pgfplotsset{compat=newest, clip bounding box=default tikz}
\usepgfplotslibrary{fillbetween}

\usepackage[capitalize]{cleveref}
% Allow floats within equations to split
\allowdisplaybreaks
\setlength\parindent{0pt}

\usepackage[square,numbers,sort]{natbib}
\usepackage{svg}

\begin{document}

% Title here
\title{\textsc{
	Notes for \textit{Randomized Sketches of Convex Programs with Sharp
	Guarantees}}}
\author{
    \textsc{Vasileios Charisopoulos}\thanks{\quad
        School of Operations Research and Information Engineering, Cornell University,
        Ithaca, NY 14850, USA.
   \email{vc333@cornell.edu}}}
\date{\today}
\maketitle

% As a general rule, do not put math, special symbols or citations
% in the abstract or keywords.
\begin{abstract}
This is a set of notes covering Pilanci \& Wainwright's \textit{Randomized
Sketches of Convex Programs with Sharp Guarantess}~\cite{PilWain15}. Emphasis
is given on the proof techniques, with additional explanations given for
several proofs.
\end{abstract}

\paragraph{Details for Corollary 2}
Consider $r := \rank(A)$. In that case, any element $v \in \mathrm{Im}(A)$ can
be written as
\[
    v = \sum_{i=1}^{r} \lambda_i a_i, \; \lambda_i \in \Rbb,
    \; a_i^\top a_j = \begin{cases}
        1, & i = j \\
        0, & i \neq j
    \end{cases}.
\]
In other words, $\set{a_i}_{i=1}^r$ is an orthonormal basis of $A$. We can thus
write
\begin{align*}
    \sup_{u \in \Rbb^d} \frac{\abs{\ip{Au, g}}}{\norm{Au}_2} &=
    \sup_{\bm{\lambda}} \frac{\abs{\sum_{i=1}^r \lambda_i a_i^\top g}}{
        \norm{\sum_{i=1}^r \lambda_i a_i}_2} \\
    &= \sup_{\lambda}
        \abs{\sum_{i=1}^r \frac{\lambda_i}{\norm{\lambda}_2} a_i^\top g} \\
    &= \sup_{\eta \in \Sbb^{\rank(A) - 1}} \abs{\eta^\top \tilde{g}}
\end{align*}
Notice that $a_i^\top g \perp a_j^\top g$, since orthogonal transformations of
Gaussian variables lead to independent gaussian variables. Therefore,
$\tilde{g} \sim \cN(0, I_r)$ itself. From the above:
\begin{align*}
    \expec{\sup_{u \in \Rbb^d} \frac{\abs{\ip{Au, g}}}{\norm{Au}_2}} &=
    \expec{\sup_{\eta \in \Sbb^{\rank(A) - 1}} \abs{\eta^\top \tilde{g}}} \leq
    \expec{\norm{\tilde{g}}_2} =
    \sqrt{\rank(A)}.
\end{align*}

\paragraph{Details for Corollary 3a}
Note the following: when $X_i \sim \cN(0, \sigma_i^2)$, it holds
that~\cite[Eq. (3.13)]{LedTal13}:
\begin{equation}
    \expec{\max_{i \in [n]} \abs{X_i}} \leq 3 \sqrt{\log n} \max_{i \in [n]}
    \expec{X_i^2}^{1/2}
    \label{eq:expected_sup_gaussian}
\end{equation}

We have the following characterization for the tangent cone, when the support of
$\bar{x}$ is $S$:
\[
    T_{\set{z \in \Rbb^d: \norm{z}_1 \leq r}}(\bar{x})
    = \set{\Delta: \ip{\Delta_S, \mathrm{sign}(\bar{x}_S)}
           + \norm{\Delta_{S^c}}_1 \leq 0}
\]
An immediate consequence, lower bounding the above by the C-S inequality, is:
\begin{equation}
    \norm{\Delta_{S^c}}_1 \leq \norm{\Delta_S}_2 \norm{\mathrm{sign}(\bar{x}_S)}
    \leq \sqrt{\abs{S}} \norm{\Delta_S}_2.
    \label{eq:delta_ineq}
\end{equation}
Observe the next chain of inequalities:
\begin{align*}
    \norm{\Delta}_1 &= \norm{\Delta_S}_1 + \norm{\Delta_{S^c}}_1
    \overset{\norm{z}_1 \leq \sqrt{d} \norm{z}_2}{\leq}
        \sqrt{\abs{S}} \norm{\Delta_S}_2 + \norm{\Delta_{S^c}}_1 \\
    &\overset{\eqref{eq:delta_ineq}}{\leq}
        2 \sqrt{\abs{S}} \norm{\Delta_S}_2 \leq 2 \sqrt{\abs{S}} \norm{\Delta}_2
\end{align*}

\bibliographystyle{plain}
\bibliography{references}

\end{document}
