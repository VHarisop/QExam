\documentclass[11pt]{beamer}

\usepackage{default}
\usepackage{../latex-macros/macros/latex-macros}
\usefonttheme[onlymath]{serif}

\mode<presentation>
{
    \usetheme
    [navbar=true,colorblocks=true,pagenumbers=true]{Cornell}
}

\begin{document}

\title{Randomized Sketches of Convex Programs with Sharp Guarantees}
\frame{\titlepage}
\frame{\tableofcontents}

%% INTRODUCTION: Problem motivation, difficulties and prior art
\section{Introduction}
\begin{frame}{Overview: a blessing...}
    % Convex programs nice, forms with closed solutions, or iterative methods
    % with efficient steps that can be computed analytically
    \begin{itemize}
    \item<1-> Convex Optimization is a fundamental tool in engineering, 
    statistics, and other disciplines.
    \item<2-> Attractive property: convex programs can be solved to global
    optimality (in practice, $\epsilon$-close to the global optimum)
    \item<3-> Plethora of theoretical (e.g. convergence, optimality conditions) 
    and practical results (e.g. algorithms, accelerated methods).
    \end{itemize}
\end{frame}
%
\begin{frame}{...and a Curse (of Dimensionality)}
    While many convex problems can be solved in polynomial time, \textit{not 
    all polynomials were created equal}.
    \linebreak
    \begin{block}<2->{An example: linear regression}
        Noisy linear measurements:
        \begin{align*}
            y_i &= \ip{a_i, x^*} + \eta_i, \; i = 1, \dots, n.
        \end{align*}
        Convex program to find least squares estimate:
        \begin{align*}
            \mbox{Minimize } & \norm{Ax - y}_2^2
        \end{align*}
        Closed-form solution, but requires matrix inversion 
        ($\sim \mathcal{O}(n^3)$)
    \end{block}
\end{frame}


\begin{frame}{More generally: statistical estimation}
    \begin{itemize}
    \item Parameter estimation + prior information about parameter
    \item Low dimensional spaces: sparse vectors, low rank matrices, etc.
    \item The convex optimization way: relax constraints.
    \end{itemize}
    \begin{block}{Examples}
        \begin{enumerate}
            \item $s$-sparse vectors $\Rightarrow$ $\norm{x}_{1} \leq s$ 
            (also known as \textit{basis pursuit}) \\
            \item $\rank(A) \leq r$ $\Rightarrow$ $\norm{A}_{*} \leq r$ 
            (nuclear norm regularization)
        \end{enumerate}
    \end{block}
    In the examples above, the ambient space of the relaxation might be
    too large!
\end{frame}

\section{A sketchy trick: random projections}
%% Overview of technique
\begin{frame}{Random projections}
    %TODO: Cite JL
    Random projections go back at least as far as 1984: Johnson and 
    Lindenstrauss showed that we can project a set of $m$ points
    in a subspace of dimension $\sim \frac{\log m}{\epsilon^2}$ without 
    distorting the distances between them more than $\epsilon$.
\end{frame}

%% Fundamental dependence
\begin{frame}{Key quantities}
    Sketching fundamentally relies on a few key quantities:
    \only<1>{
    \begin{block}{Tangent cone $\cK$}
        Given a constraint set $\cC \subseteq \Rbb^d$, the cone of all feasible 
        directions from the optimum $x^* \in \cC$ is defined as
        \[
            \cK := \mathrm{clconv}\set{
                z \in \Rbb^d \mmid z = t (x - x^*), \; t \geq 0, \;
                x \in \cC
            }
        \]
    \end{block}
    Since the objective function is $\norm{Ax - y}_2^2$, we need to examine the
    \textbf{transformed} cone:
    \[
        A \cK := \set{A z \in \Rbb^n \mmid z \in \cK}
    \]}
    \only<2>{
        \begin{block}{Gaussian width}
            Given a set $S \subseteq \Rbb^n$, we define its \textbf{Gaussian 
            width} as
            \[
                \mathbb{W}(S) := \expec[g]{\sup_{z \in S} 
                \abs{\ip{g, z}}}, \quad g \sim \cN(0, I_n)
            \]
        \end{block}
        \textbf{Interpretation}: in a probabilistic scenario, sets with large 
        gaussian width will exhibit more degrees of freedom.
        %TODO: find an example for the above
    }
\end{frame}

\begin{frame}{Key quantities}
    
\end{frame}

%% MAIN: Presentation of main result(s), what is guaranteed and what's not
\section{Main result}

\end{document}
